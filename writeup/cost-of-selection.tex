\documentclass{article}
\usepackage{amsmath,amssymb}
\usepackage[hidelinks]{hyperref}
\usepackage{xcolor}
\pagestyle{empty}
\linespread{1.15}

\usepackage{natbib}

\title{Biobank Supplement}
\author{kkasimat}
\date{April 2020}

\begin{document}

\noindent
\textbf{Supplemental File 9. Estimating the sex-specific mortality cost}

\noindent
In the model we focus on in this paper -- sex-dependent viability selection on an autosomal variant --
any differences in allele frequencies among the adult females and adult males
must happen due to differential, genotype-dependent mortality during that generation,
as discussed in \citet{Kasimatis:2019cs}.
How much ``excess'' mortality -- genotype-dependent mortality --
must happen to cause a frequency difference of $f$?
Calculations of the ``cost of selection'' \citep{Haldane:1957ka} are standard,
but we provide the argument here for clarity,
and to disambiguate the rate of ``excess mortality'' from the selection coefficient,
a related but distinct concept.

Suppose that before mortality selection, the focal allele is at frequency $p$ in both sexes.
Then suppose that selection changes the frequency in some population (e.g., the ``population'' of males)
to $p - f$.
How many ``selective deaths'' was this?
If we are to remove a proportion $u$ of the focal alleles (leaving all of the other allele),
then the frequency of the focal allele after selection is
\begin{align*}
    \frac{p (1-u)}{p (1-u) + (1 - p)} = \frac{p (1-u)}{1 - u p},
\end{align*}
and so if the frequency after selection is $p-f$ we need that
\begin{align*}
    \frac{p (1-u)}{(1 - u p)} = p - f  .
\end{align*}
Rearranging this equation,
we obtain that
% \begin{align*}
%    p - p u = p - u p^2 - (1 - u p) f , \\
%    f = u (pf + p - p^2) = u p (f + 1 - p) ,
% \end{align*}
% and so
\begin{align*}
    u = \frac{f}{p (f + 1 - p)} .
\end{align*}

% For $f$ small, the denominator is nearly $p (1 - p)$, which is at most $1/4$,
% and so $u$ is at least $4f$.
% In other words, the proportion \emph{of a particular allele} that must die to move the frequency by $f$ is at \emph{least} $4 f$.

Here, $u$ is the proportion of individuals \emph{with the focal allele} that suffered selective mortality, but we want to know the proportion of the total population.
Since in a population of size $N$, the number of selective deaths is $u p N$,
the proportion of the total population that is removed by selection is $u p$,
which in terms of $p$ and $f$ is
\begin{align*}
    u p = \frac{f}{f + 1 - p} .
\end{align*}
At $p = 1/2$ and with $f$ small, $u$ is close to $2 f$,
while at small $p$, $u$ is close to $f$,
and increases with increasing $p$.
Thus, the proportion \emph{of the population} that must die to move the frequency by $f$
is between $f$ and $2 f$.
To make a frequency difference of $f$ between the sexes,
the frequency in each sex could change by $f/2$ in opposite directions,
and so at least $f/2$ of each of the sexes (and thus, the total population)
would have to be removed by selection.

%This is where your bibliography is generated. Make sure that your .bib file is actually called library.bib
\bibliography{ref}

%This defines the bibliographies style. Search online for a list of available styles.
\bibliographystyle{abbrvnat}

\end{document}
